\begin{table}[H]
\centering
\caption{\label{arrest_prob}Effect of ShotSpotter Enactment on 911 Arrest Likelihood and Final Dispositions (Gardner 2021)}
\centering
\begin{threeparttable}
\fontsize{11}{13}\selectfont
\begin{tabular}[t]{lcccccc}
\toprule
\multicolumn{1}{c}{ } & \multicolumn{3}{c}{911 Arrests} & \multicolumn{3}{c}{Most Frequent Misc. 911 Dispositions} \\
\cmidrule(l{3pt}r{3pt}){2-4} \cmidrule(l{3pt}r{3pt}){5-7}
\multicolumn{1}{c}{ } & \multicolumn{1}{c}{\makecell[c]{Total\\Arrests}} & \multicolumn{1}{c}{\makecell[c]{Gun\\Arrests}} & \multicolumn{1}{c}{\makecell[c]{Non-Gun\\Arrests}} & \multicolumn{1}{c}{\makecell[c]{Other\\Police Service}} & \multicolumn{1}{c}{\makecell[c]{No\\Person Found}} & \multicolumn{1}{c}{\makecell[c]{Peace\\Restored}} \\
\cmidrule(l{3pt}r{3pt}){2-2} \cmidrule(l{3pt}r{3pt}){3-3} \cmidrule(l{3pt}r{3pt}){4-4} \cmidrule(l{3pt}r{3pt}){5-5} \cmidrule(l{3pt}r{3pt}){6-6} \cmidrule(l{3pt}r{3pt}){7-7}
  & (1) & (2) & (3) & (4) & (5) & (6)\\
\midrule
ShotSpotter Activated & -0.267*** & -0.520*** & -0.241*** & 1.316* & 1.151*** & -0.482*\\
 & (0.060) & (0.176) & (0.065) & (0.686) & (0.411) & (0.261)\\
Observations & 3,582,528 & 317,937 & 3,264,591 & 3,582,528 & 3,582,528 & 3,582,528\\
Mean of Dependent Variable & 2.502 & 3.421 & 2.413 & 41.810 & 21.341 & 6.062\\
FE: Day-by-Month-by\_year & X & X & X & X & X & X\\
\midrule
\addlinespace
FE: Distrct & X & X & X & X & X & X\\
FE: Call-Type & X & X & X & X & X & X\\
FE: Hour-of-Day & X & X & X & X & X & X\\
Gardner (2021) & X & X & X & X & X & X\\
\bottomrule
\end{tabular}
\begin{tablenotes}
\item \textit{Note: } 
\item * p < 0.1, ** p < 0.05, *** p < 0.01
\item Standard errors are clustered by district. All                      coefficient estimates are in percentages. All estimates                      are computed using the Gardner (2021) estimator.                       The dependent variable in Columns 1-3 is an indicator equal to one if a 911 call resulted in an arrest.                      The dependent variable in Columns 4-6 is an indicator equal to one if a 911 call resulted in                       Other Police Service (Column 4), No Person Found (Column 5), or Peace Restored (Column 6).                      Column 1 reports the estimates using the entire sample.                  Columns 2 and 3 subset Column 1 by gun-related and non-gun-related 911 calls.                  Gun-related crimes are those corresponding to the following                  911 code descriptions: `person with a gun',                  `shots fired', or `person shot'.                   Columns 4-6 report the three most frequent 911 final dispositions: Other Police Service, No Person Found,                   and Peace Restored. The final disposition is the final result of                  what happened on the 911 call.                   Wild cluster bootstrap p-values using 999 replications are also reported                  since the number of clusters (22) is below the threshold of 30 put forth in                  Cameron et al. (2008).                  
\end{tablenotes}
\end{threeparttable}
\end{table}

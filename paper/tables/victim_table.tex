\begin{table}[H]

\caption{\label{victim_table}Effect of ShotSpotter Implementation on Likelihood of 911 Victim Injury (OLS)}
\centering
\begin{threeparttable}
\fontsize{11}{13}\selectfont
\begin{tabular}[t]{>{\raggedright\arraybackslash}p{8cm}ccc}
\toprule
\multicolumn{1}{c}{ } & \multicolumn{3}{c}{Likelihood of Victim Injury} \\
\cmidrule(l{3pt}r{3pt}){2-4}
\multicolumn{1}{c}{ } & \multicolumn{1}{c}{Pooled} & \multicolumn{1}{c}{Gun Dispatch} & \multicolumn{1}{c}{Non-Gun Dispatch} \\
\cmidrule(l{3pt}r{3pt}){2-2} \cmidrule(l{3pt}r{3pt}){3-3} \cmidrule(l{3pt}r{3pt}){4-4}
  & (1) & (2) & (3)\\
\midrule
ShotSpotter Activated & -0.062 & -0.422* & -0.007\\
 & (0.051) & (0.211) & (0.054)\\
Mean of Dependent Variable & 2.990 & 4.185 & 2.874\\
Observations & 3,582,560 & 317,937 & 3,264,623\\
Wild Cluster Bootstrap P-Value & 0.245 & 0.067 & 0.895\\
\midrule
\addlinespace
FE: Day-by-Month-by-Year & X & X & X\\
FE: District & X & X & X\\
FE: Call-Type & X & X & X\\
FE: Hour-of-Day & X & X & X\\
\bottomrule
\end{tabular}
\begin{tablenotes}
\item \textit{Note: } 
\item * p < 0.1, ** p < 0.05, *** p < 0.01
\item Standard errors are clustered by district. All coefficient                      estimates are in percentages.                      The main variable is the probability of a victim being                      injured during a 911 call dispatch.                      The Pooled column reports estimates using the entire sample of Priority 1                      dispatches.                      Gun Dispatch (Column 2) is restricted to only gun-related 911 call dispatches which                      have the following 911 code descriptions:                      `person with a gun',                  `shots fired', or `person shot'. Non-Gun Dispatch (Column 3) are all other                      911 call dispatches that are not related to gun descriptions. In all columns the preferred specification is estimated using                      OLS. Wild cluster bootstrap p-values using 999 replications are also reported                  since the number of clusters (22) is below the threshold of 30 put forth in                  Cameron et al. (2008).                                    
\end{tablenotes}
\end{threeparttable}
\end{table}
